\documentclass[12pt,a4paper]{article}
\usepackage{geometry}
\geometry{margin=1in}
\usepackage{graphicx}
\usepackage{setspace}
\usepackage{titling}
\usepackage{titlesec}
\usepackage{tabularx}
\usepackage{booktabs}
\usepackage{enumitem}
\usepackage{microtype}
\usepackage{newtxtext,newtxmath}
\usepackage[hidelinks]{hyperref}

\titlespacing{\section}{0pt}{0.5em}{0.2em}
\setlength{\parskip}{0.2em}
\setlength{\parindent}{0pt}
\pagenumbering{gobble}

\begin{document}

% --------- COVER PAGE ----------
\begin{titlepage}
    \begin{center}
        \vspace*{0.4cm}
        \includegraphics[width=0.19\textwidth]{Logo_of_Premier_University_(PU).png}\\[0.8cm]
        
        {\fontsize{28}{34}\selectfont \textbf{Premier University}}\\[0.15cm]
        {\fontsize{17}{21}\selectfont Chattogram}\\[0.6cm]

        {\Large \textit{Project Proposal}}\\[0.2cm]

        {\fontsize{18}{24}\selectfont \textbf{Design and Simulation of an IPv6 Smart City IoT Network}}\\[0.08cm]
        {\fontsize{13}{16}\selectfont \textbf{with Quality of Service and Resilient Routing}}\\[0.7cm]

        {\large \textbf{Submitted by}}\\[0.08cm]
\rule{0.21\textwidth}{0.7pt}\\[0.12cm]

\renewcommand{\arraystretch}{1.2}
\begin{tabular}{p{6.5cm} l}
    \textbf{Name} & \textbf{ID} \\
    MD Nishadul Islam Chy Shezan & 0222220005101014 \\
    MD Sakib & 0222220005101019 \\
    Rimjhim Dey & 0222220005101039 \\
\end{tabular}\\[0.6cm]

\begin{center}
    \textbf{Section:} A \\[0.1cm]
    \textbf{Batch:} 42 \\[0.1cm]
    \textbf{Session:} Spring 2025
\end{center}

        {\large \textbf{Submitted to}}\\[0.08cm]
        \rule{0.17\textwidth}{0.7pt}\\[0.11cm]
        {\textbf{Dr. Shahid Md. Asif Iqbal}}\\
        \textit{Professor}\\
        Department of Computer Science and Engineering\\
        \textbf{\textit{Associate Dean}}\\
        Faculty of Engineering\\[0.35cm]

        \vfill
        {\normalsize August 2025}
    \end{center}
\end{titlepage}

% --------- CONTENTS PAGE ----------
\newpage
\begin{center}
    {\Huge \textbf{Contents}}
\end{center}
\vspace{1.5em}

\tableofcontents
\newpage

\pagenumbering{arabic}

% --------- MAIN CONTENT ----------

\section*{Introduction}
\phantomsection
\addcontentsline{toc}{section}{Introduction}
Modern cities deploy numerous sensors to collect information about traffic, air quality, and infrastructure usage. However, most networks still rely on IPv4, which struggles with address limitations and lacks proper support for IoT deployments. This project proposes designing a smart city network in Cisco Packet Tracer using IPv6 for unlimited addressing, Quality of Service (QoS) for prioritizing emergency traffic, and edge computing for faster local processing. 

The network will demonstrate how these technologies can improve city sensor systems, especially during failures. Current IPv4 limitations include address exhaustion requiring complex NAT, lack of native IoT support, and centralized processing bottlenecks that our hierarchical IPv6 architecture with distributed fog computing will address.

\section*{Objectives}
\phantomsection
\addcontentsline{toc}{section}{Objectives}
\begin{enumerate}[label=2.\arabic*, nosep]
    \item Design a smart city network in Cisco Packet Tracer with access, distribution, and core layers
    \item Implement IPv6 everywhere to eliminate address exhaustion
    \item Configure Quality of Service to prioritize emergency traffic
    \item Deploy edge (fog) routers for local sensor data processing
    \item Test network failover using HSRP and EIGRP for IPv6
    \item Separate IoT, public Wi-Fi, and admin devices using VLANs and ACLs
    \item Create email alerts through SMTP for critical events
\end{enumerate}

\section*{Scope}
\phantomsection
\addcontentsline{toc}{section}{Scope}
This project covers the design, implementation, and testing of a smart city IoT network prototype in Cisco Packet Tracer. It includes:
\begin{itemize}[nosep]
    \item Three-tier network architecture with IPv6 addressing
    \item Security implementation through VLANs and ACLs
    \item Basic network services (SMTP, HTTP)
    \item Simulated IoT devices (traffic sensors, air quality monitors, smart bins)
    \item Failover testing and QoS validation
\end{itemize}

The project excludes physical layer considerations, advanced security features like IPSec, complex application development, and real-time analytics platforms to maintain focus on core networking concepts.

\section*{Tools and Technologies}
\phantomsection
\addcontentsline{toc}{section}{Tools and Technologies}
\textbf{Primary Platform:} Cisco Packet Tracer 8.2.x with IoT device templates

\textbf{Networking Protocols:}
\begin{itemize}[nosep]
    \item IPv6 (RFC 8200) for addressing
    \item EIGRP for IPv6 routing
    \item HSRP for redundancy
    \item IEEE 802.1Q for VLANs
    \item DiffServ for QoS
\end{itemize}

\textbf{Services:} DHCPv6, SLAAC, SMTP, HTTP, ICMPv6

\textbf{Security:} Extended ACLs, VLAN segmentation, Port security

\section*{Key Features}
\phantomsection
\addcontentsline{toc}{section}{Key Features}
\begin{enumerate}[label=5.\arabic*, nosep]
    \item \textbf{IPv6 Implementation:} Native IPv6 with /48 for sites and /64 for subnets, eliminating NAT complexity
    \item \textbf{Quality of Service:} Four-tier traffic classification (Network Control, Emergency Services, Standard IoT, Best Effort)
    \item \textbf{Edge Computing:} Fog routers near sensors for local processing and reduced latency
    \item \textbf{Redundancy:} HSRP with sub-second failover and EIGRP for dynamic routing
    \item \textbf{Security:} VLAN isolation between device types with ACL-based access control
    \item \textbf{Monitoring:} SMTP alerts for critical events and HTTP dashboard for status
\end{enumerate}


\section*{Project Timeline}
\phantomsection
\addcontentsline{toc}{section}{Project Timeline}

\vspace{1em}
\renewcommand{\arraystretch}{1.2}
\setlength{\extrarowheight}{4pt}

\begin{center}
\begin{tabularx}{\textwidth}{
    |>{\raggedright\arraybackslash}p{2.4cm}
    |>{\raggedright\arraybackslash}p{4.2cm}
    |>{\raggedright\arraybackslash}X|
}
\hline
\textbf{Weeks} & \textbf{Phase} & \textbf{Key Activities} \\
\hline
Weeks 1--2 & Literature Review & Research smart city networks and IPv6 deployments \\
\hline
Weeks 3--4 & Project Planning & Finalize objectives and submit proposal \\
\hline
Weeks 5--6 & Network Design & Create topology in Packet Tracer \\
\hline
Weeks 7--8 & Configuration & Implement IPv6, VLANs, and QoS \\
\hline
Weeks 9--10 & Testing & Validate failover and security features \\
\hline
Week 11 & Documentation & Compile results and prepare report \\
\hline
Week 12 & Submission & Submit project and presentation \\
\hline
\end{tabularx}
\end{center}

\section*{Expected Outcomes}
\phantomsection
\addcontentsline{toc}{section}{Expected Outcomes}
\begin{enumerate}[label=7.\arabic*, nosep]
    \item A functional smart city network supporting 15+ IoT devices
    \item Demonstrated sub-second failover during router failures
    \item Validated QoS ensuring emergency traffic priority
    \item Complete network documentation with configuration templates
    \item Hands-on experience with enterprise networking technologies
    \item Reference architecture for real-world smart city deployments
\end{enumerate}

The project will validate that IPv6 and edge computing effectively address smart city requirements while providing practical experience for team members.

\section*{Complex Engineering Problem Attributes}
\phantomsection
\addcontentsline{toc}{section}{Complex Engineering Problem Attributes}

\begin{center}
\renewcommand{\arraystretch}{1.2}
\begin{tabularx}{\textwidth}{|p{4.2cm}|X|}
\hline
\textbf{Attribute} & \textbf{How Project Meets It} \\
\hline
Conflicting Requirements & Balance emergency priority with fair public access \\
\hline
Multiple Stakeholders & IoT sensors, city staff, admin PCs, and public users share the network \\
\hline
Depth of Analysis & Requires IPv6, VLANs, ACLs, routing, and QoS knowledge \\
\hline
Extensive Knowledge & Combines networking, security, and IoT concepts \\
\hline
Interdependence & All components depend on each other to function \\
\hline
Public Impact & Improves city operations and emergency response \\
\hline
Innovation & Combines IPv6 and edge computing for IoT networks \\
\hline
\end{tabularx}
\end{center}

\section*{Limitations}
\phantomsection
\addcontentsline{toc}{section}{Limitations}
\begin{itemize}[nosep]
    \item \textbf{Simulation constraints:} Packet Tracer cannot replicate all real-world network behaviors
    \item \textbf{Scale:} Limited to 15-20 devices versus thousands in actual cities
    \item \textbf{Wireless:} Cannot test real interference and propagation issues
    \item \textbf{Services:} Basic SMTP/HTTP compared to production systems
    \item \textbf{Security:} No encryption or advanced threat protection
\end{itemize}

Despite these limitations, the project effectively demonstrates core concepts that scale to production environments.

\section*{Conclusion}
\phantomsection
\addcontentsline{toc}{section}{Conclusion}
This project addresses critical networking challenges faced by modern smart cities through a comprehensive IPv6-based IoT network with QoS and edge computing. The implementation demonstrates practical solutions for address exhaustion, traffic prioritization, and network resilience. By combining current technologies with proper design, we show how cities can build efficient sensor networks that improve urban operations and emergency response.

The hands-on experience gained and documented best practices will serve as valuable resources for future smart city deployments. While simulation limitations exist, the core concepts and architectures directly apply to real-world implementations, contributing to the development of more sustainable and resilient urban environments.

\end{document}